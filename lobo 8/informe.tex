\documentclass[fleqn,10pt]{article}

% --- Preámbulo: Paquetes ---
\usepackage[utf8]{inputenc}
\usepackage[T1]{fontenc}
\usepackage{graphicx}
\usepackage{geometry}
\usepackage{amsmath,amssymb}
\usepackage{listings}
\usepackage{xcolor}
\usepackage{tcolorbox}
\usepackage{booktabs}
\usepackage{caption}
\usepackage{hyperref}
\usepackage{makecell}
\usepackage{enumitem}

% --- Configuración de página ---
\geometry{a4paper, margin=1in}

% --- Definiciones de colores para código ---
\definecolor{codegreen}{rgb}{0,0.6,0}
\definecolor{codegray}{rgb}{0.5,0.5,0.5}
\definecolor{codepurple}{rgb}{0.58,0,0.82}
\definecolor{backcolour}{rgb}{0.95,0.95,0.92}

% --- Estilo para código Python ---
\lstdefinestyle{pythonstyle}{
    backgroundcolor=\color{backcolour},
    commentstyle=\color{codegreen},
    keywordstyle=\color{magenta},
    numberstyle=\tiny\color{codegray},
    stringstyle=\color{codepurple},
    basicstyle=\ttfamily\footnotesize,
    breaklines=true,
    captionpos=b,
    keepspaces=true,
    tabsize=2,
    language=Python,
    showstringspaces=false,
    numbers=left,
    numbersep=5pt,
    frame=single,
    rulecolor=\color{black}
}

% --- Estilo para salida de consola o bloques de texto ---
\lstdefinestyle{outputstyle}{
    backgroundcolor=\color{backcolour},
    basicstyle=\ttfamily\footnotesize,
    breaklines=true,
    captionpos=b,
    keepspaces=true,
    numbers=none,
    frame=single,
    rulecolor=\color{black},
    showstringspaces=false
}

\begin{document}

% --- Encabezado principal ---
\begin{center}
    {\LARGE\bfseries UNIVERSIDAD NACIONAL DE SAN ANTONIO ABAD DEL CUSCO}\\[0.3cm]
    {\Large FACULTAD DE INGENIERÍA ELÉCTRICA, ELECTRÓNICA, INFORMÁTICA Y MECÁNICA}\\[0.3cm]
    {\Large ESCUELA PROFESIONAL DE INGENIERÍA INFORMÁTICA Y DE SISTEMAS}\\[1cm]
    \includegraphics[width=0.25\linewidth]{Escudo_UNSAAC.png}\\[1cm]
    {\Large\bfseries CURSO: BIOINFORMÁTICA}\\[0.3cm]
    {\Large\bfseries TRABAJO: LABORATORIO 8 }\\[0.3cm]
    {\Large\bfseries PROFESORA: MARIA DEL PILAR VENEGAS VERGARA}\\[0.3cm]
    {\Large\bfseries ALUMNO: EFRAIN VITORINO MARÍN}\\[0.3cm]
    {\Large\bfseries CÓDIGO: 160337}\\[0.3cm]
    {\Large 2025-I}
\end{center}
\newpage

% --- Contenido principal ---
\section{Actividad 1: Responder a las siguientes interrogantes}

\subsection{1. ¿Qué es alineamiento múltiple?}

El \textbf{alineamiento múltiple} es una extensión del alineamiento por pares que permite comparar simultáneamente tres o más secuencias de ADN, ARN o proteínas para identificar regiones de similitud que pueden indicar relaciones funcionales, estructurales o evolutivas.

\begin{tcolorbox}[breakable, colback=blue!5!white,colframe=blue!75!black,title=Definición Formal]
Sea $S = \{s_1, s_2, \ldots, s_k\}$ un conjunto de $k$ secuencias sobre un alfabeto $\Sigma$. Un alineamiento múltiple $A$ es una matriz donde:
\begin{itemize}
    \item Cada fila $i$ representa la secuencia $s_i$ con posibles gaps (-)
    \item Todas las filas tienen la misma longitud $L$
    \item Al remover los gaps de la fila $i$, se obtiene la secuencia original $s_i$
\end{itemize}
\end{tcolorbox}

\textbf{Características principales:}
\begin{enumerate}[label=\textbf{\alph*)}]
    \item \textbf{Conservación evolutiva}: Identifica regiones conservadas entre especies
    \item \textbf{Análisis funcional}: Revela sitios activos y dominios funcionales
    \item \textbf{Predicción estructural}: Ayuda a predecir estructuras secundarias y terciarias
\end{enumerate}

\subsection{2. ¿Qué diferencias tienen los algoritmos de alineamiento múltiple, frente a alineamiento local y global?}

\begin{table}[htbp]
\centering
\caption{Comparación entre tipos de alineamiento}
\begin{tabular}{@{}lccc@{}}
\toprule
\textbf{Característica} & \textbf{Global} & \textbf{Local} & \textbf{Múltiple} \\
\midrule
\makecell{Número de \\ secuencias} & 2 & 2 & $\geq 3$ \\
\makecell{Cobertura de \\ la secuencia} & Completa & Parcial & Variable \\
\makecell{Complejidad \\ temporal} & $O(mn)$ & $O(mn)$ & $O(L^k)$ \\
\makecell{Algoritmo \\ principal} & \makecell{Needleman-\\Wunsch} & \makecell{Smith-\\Waterman} & \makecell{ClustalW, MUSCLE,\\T-Coffee} \\
\makecell{Función \\ objetivo} & \makecell{Maximizar similitud\\global} & \makecell{Encontrar regiones\\similares} & \makecell{Optimizar suma\\de pares} \\
\bottomrule
\end{tabular}
\end{table}

\begin{tcolorbox}[breakable, colback=green!5!white,colframe=green!75!black,title=Teorema de Complejidad]
\textbf{Teorema}: El problema de alineamiento múltiple óptimo es NP-completo.

\textbf{Demostración}: Se reduce del problema de la subsecuencia común más larga (LCS) para múltiples secuencias, que es conocido como NP-completo.

\textbf{Implicación}: Para $k$ secuencias de longitud promedio $n$, la complejidad es $O(n^k)$, lo que hace inviable la solución exacta para grandes valores de $k$.
\end{tcolorbox}

\textbf{Diferencias fundamentales:}

\begin{enumerate}[label=\textbf{\Roman*.}]
    \item \textbf{Dimensionalidad del problema}:
    \begin{itemize}
        \item Alineamiento por pares: Matriz 2D
        \item Alineamiento múltiple: Hipermatriz $k$-dimensional
    \end{itemize}
    
    \item \textbf{Estrategias algorítmicas}:
    \begin{itemize}
        \item \textbf{Progresivas}: Construyen el alineamiento paso a paso (ClustalW)
        \item \textbf{Iterativas}: Refinan alineamientos iniciales (MUSCLE)
        \item \textbf{Consistencia}: Maximizan la consistencia entre alineamientos por pares (T-Coffee)
    \end{itemize}
    
    \item \textbf{Función de puntuación}:
    \begin{align}
        \text{Score}_{\text{múltiple}} &= \sum_{i<j} \text{Score}(s_i, s_j) \quad \text{(Suma de pares)} \\
        \text{Score}_{\text{global}} &= \max_{A} \sum_{i=1}^{L} \sigma(A[i,1], A[i,2]) \\
        \text{Score}_{\text{local}} &= \max_{A,i,j} \sum_{k=i}^{j} \sigma(A[k,1], A[k,2])
    \end{align}
\end{enumerate}

\begin{tcolorbox}[breakable, colback=red!5!white,colframe=red!75!black,title=Limitaciones Computacionales]
\textbf{Problema}: La programación dinámica exacta para $k$ secuencias requiere:
\begin{itemize}
    \item Espacio: $O(n^k)$
    \item Tiempo: $O(kn^k)$
\end{itemize}

\textbf{Solución}: Uso de heurísticas y aproximaciones que sacrifican optimalidad por eficiencia computacional.
\end{tcolorbox}
\newpage

\section{Actividad 2: Utilizar el algoritmo Blast de NCBI para alinear las secuencias de la tabla 1}

\begin{table}[htbp]
\centering
\caption{Secuencias para alineamiento múltiple}
\begin{tabular}{|p{2cm}|p{12cm}|}
\hline
\textbf{Secuencia} & \textbf{Secuencia completa} \\
\hline
Secuencia 1 & \texttt{MMALGRAFAIVFCLIQAVSGESGNAQDGDLEDADADDHSFWCHSQLEVDGSQHLLTCAFNDSDINTANLEFQI CGALLRVKCLTLNKLQDIYFIKTSEFLLIGSSNICVKLGQKNLTCKNMAINTIVKAEAPSDLKVVYRKEANDF LVTFNAPHLKKKYLKKVKHDVAYRPARGESNWTHVSLFHTRTTIPQRKLRPKAMYEIKVRSIPHNDYFKGFWS EWSPSSTFETPEPKNQGGWDPVLPSVTILSLFSVFLLVILAHLVLWKRIKPVVWPSLPDHKKTLEQLCHKPKT SLNVSFNPESFLDCQIHEVKGVEARDEVESFLPNDLPAQPEELETNIPQGHRAAVHSANRSPETSVSPPLNKL RESPLRCLATCNAPPLLSSRSPDYRDGDRNRPPVYQDLLPNSGNTNVPVPVPQPLPFQSGILIPVSQRQPIST SSVLNQEEAYVTMSSFYQNK} \\
\hline
Secuencia 2 & \texttt{NRGETGAPAGPRGPAGPAGSSGKDGVGGLPGPIGPPSPRGRTGDIGPAGPPGTPGPPGPPGPPGGGFDFSFVA QPSQEKAPDPFRHYRADDANVARDRDLEVDTTLKSLSQQKDLAIENIRSPEGTKKDPARSCRDLKMCHPEWKS GEYFVDPNQGCDEDAVKVYCNMETGETCVYPTQANIPQKNWYTSKNAKDKKHVWFGETMSDGFQFEYGGEGSD AADVNIQLTFLRLMATEASQNITYHCKNSIAYMDQQAGNLKKALLLQGSNEIEIRAEGNSRFTYSEETEDGYT RHTGAWGKTVIDADYKTTKTSRLPIIDIAPMDVGAPDQEFGIDVGP} \\
\hline
Secuencia 3 & \texttt{MSFSRRPKITKSDIVDTVYFQISLNIRNNNLKLEKKKIRLVIDAFFEELKGNLALNNVIEFRSFGTEVRKRK GRLNPRSEYKVLHDHVAYYHTYQGFPSHSCHIPKDLALFTFYEIWVEATNRRGSARSDVLTLDEVDTVTTDPPP EVHVSRVGGLEDQLSVRWVSPPALKLKERVWGIKG} \\
\hline
Secuencia 4 & \texttt{TGGGATGATTCCACACCCGCGCCCGGCACCCGCGTCCGCGCCGTAGCCATCAACAAGCAGTCACAGCACATGA CGGAGGTTGTGAGGCGCTGCCTCCACGATGAGCGCTACTCAGATAGCGATA} \\
\hline
Secuencia 5 & \texttt{ACTATAAAGGCGTCAAGCCGTGTTCTAGATAATAATAAGTATTGGGCAACTTATTAGTCTCCGGTCCAACAAC CTGAACGGATTTGATGAAATGGGC} \\
\hline
Secuencia 6 & \texttt{ATATTGGTGTGTGAGGCGTTATAATTCCAAGAAGCAAGTGAACTTTGATAGAACAGGTCTTCGGCTTCGTGGT TAAACTTGTCCAAATGTGAGGCGGCCTGTTCCTCAATGGTGGACTGAGCAGCAGTTACAGCAACAAGGCTGAG AAGGAGCCAGGAAGAGCTTGACATCGTCGCCTCCACAGCCAAGATCACATCCACTGAATGACTTTCCCTAGAC TAAAACCTCCTCATGAGATTTTCTCTCTTATCAGCCTTTGAACTTGGGTTGGGCGCTGAGCAGGAAAGACCAA AAAAAGAAAAAGAAGAAGAACACAGTAAACAATCTGCTGAGCCAATATAAAGTTCATCCTGGAGAGGACAGAT ATGTAACAGATTTTAGAATAATTTTTTAAAGTGAATCAAATAAGAATACGTTATTCTTTAATCCTAGAGAACC TTATCACCTCCGGTCAAATCTCAGGTATCTTGGGGCCCGAGGGCCCAGTATGTCCACGATGCATACCTGCAGA TAAAGATCGCGTCTTGGGTGAGGGCTCCGCGTTATCAATTGGGTCCCCGAACTGGGAAGACTGAAATGCTAGT TTGCGAGTATATAAGAAGACCTCTATAGTGCGAGTATAAGATCATCGAAGAAGGTCGGCGGCTTGTCCGTTTA CTCACTGCTCTTGTGACATAGTAACAACAAGTAACCTCGCCTTAATTGACTGAAGGCATTCCTCGTGCAGTGT GAGGCG} \\
\hline
\end{tabular}
\end{table}


\subsection*{Resultados de BLAST para las secuencias de la tabla}




\end{document}
