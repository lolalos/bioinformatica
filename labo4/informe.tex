\documentclass[fleqn,10pt]{article}
\usepackage[utf8]{inputenc}
\usepackage{graphicx} % Para incluir imágenes
\usepackage{geometry} % Para ajustar los márgenes
\usepackage{multicol} % Para usar múltiples columnas
\usepackage{amsmath} % Para fórmulas matemáticas
\geometry{a4paper, margin=1in}

% Información del título
\title{\textbf{\Huge BÚSQUEDA DE EXONES}}

\begin{document}

% Primera hoja con la información de la universidad y los datos del curso
\begin{titlepage}
    \centering
    \vspace*{1cm} % Espacio al principio de la hoja
    {\LARGE \textbf{UNIVERSIDAD NACIONAL DE SAN ANTONIO ABAD DEL CUSCO} \par}
    \vspace{0.5cm}
    {\Large FACULTAD DE INGENIERÍA ELÉCTRICA, ELECTRÓNICA, INFORMÁTICA Y MECÁNICA \par}
    \vspace{0.5cm}
    {\Large ESCUELA PROFESIONAL DE INGENIERÍA INFORMÁTICA Y DE SISTEMAS \par}
    \vfill
    \includegraphics[width=0.25\linewidth]{Escudo_UNSAAC.png}  
    
    % Datos del curso y trabajo
    {\Large \textbf{CURSO: BIOINFORMÁTICA} \par}
    \vspace{0.5cm}
    {\Large \textbf{TRABAJO: BÚSQUEDA DE EXONES} \par}
    
    \vspace{0.5cm}
    {\Large \textbf{PROFESORA: MARIA DEL PILAR VENEGAS VERGARA} \par}
    \vspace{1cm}

    % Datos del alumno
    {\Large \textbf{ALUMNO: EFRAIN VITORINO MARÍN } \par}
    \vspace{0.5cm}
    {\Large \textbf{CÓDIGO: 160337} \par}
    \vfill

    % Semestre
    {\Large \textbf{2025-I} \par}
    \vspace{1cm}
\end{titlepage}

\flushbottom
\maketitle
\thispagestyle{empty}

\section*{Actividad 1: Responder a las siguientes interrogantes}

\begin{enumerate}
    \item \textbf{¿Qué es exón?}
    
    Un \textbf{exón} es una secuencia de nucleótidos dentro de un gen que contiene información codificante, es decir, que se traduce en proteínas. Durante el proceso de transcripción y maduración del ARN, los exones se conservan y se empalman entre sí para formar el ARN mensajero (ARNm) funcional:
    \[
    \text{Gen} = \underbrace{\text{Exón}_1}_{\text{codificante}} + \underbrace{\text{Intrón}_1}_{\text{no codificante}} + \underbrace{\text{Exón}_2}_{\text{codificante}} + \cdots
    \]
    \[
    \text{ARNm final} = \text{Exón}_1 + \text{Exón}_2 + \cdots
    \]
    
    \item \textbf{¿Qué es intrón?}
    
    Un \textbf{intrón} es una secuencia no codificante dentro de un gen. Los intrones son transcritos junto con los exones en el ARN primario, pero son eliminados durante el proceso de empalme (\textit{splicing}):
    \[
    \text{ARN primario} = \text{Exón}_1 + \text{Intrón}_1 + \text{Exón}_2 + \cdots
    \]
    \[
    \text{Empalme:} \quad \text{Intrón}_1 \rightarrow \text{eliminado}
    \]
    \[
    \text{ARNm final} = \text{Exón}_1 + \text{Exón}_2 + \cdots
    \]
    
    \item \textbf{¿Cómo se detectan exones?}
    
    Los exones se detectan mediante técnicas bioinformáticas que comparan secuencias de ADN o ARN con exones de referencia. Se utilizan algoritmos de comparación de patrones como Levenshtein y Hamming:
    \[
    d_H(s_1, s_2) = \sum_{i=1}^{n} [s_1(i) \neq s_2(i)]
    \]
    \[
    d_L(s_1, s_2) = \text{mínimo número de operaciones para transformar } s_1 \text{ en } s_2
    \]
    También se emplean alineamientos de secuencias y anotaciones genómicas.
    
    \item \textbf{¿Cómo se detectan intrones?}
    
    Los intrones se detectan indirectamente, identificando primero los exones. Las regiones entre exones que no coinciden con ningún exón de referencia se consideran intrones:
    \[
    \text{Gen} = \text{Exón}_1 + \boxed{\text{Intrón}_1} + \text{Exón}_2
    \]
    Herramientas de análisis genómico y secuenciación de ARN permiten identificar los puntos de corte (\textit{splicing sites}).
    
    \item \textbf{¿Qué es exón de referencia?}
    
    Un \textbf{exón de referencia} es una secuencia conocida y validada experimentalmente que representa un exón típico de un gen. Se utiliza como patrón para identificar exones similares:
    \[
    \text{Secuencia candidata} \approx \text{Exón de referencia}
    \]
    Su uso es fundamental para detectar mutaciones, inserciones o deleciones mediante comparación con este modelo estándar.
\end{enumerate}

\end{document}
