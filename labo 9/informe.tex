\documentclass[fleqn,10pt]{article}

% --- Preámbulo: Paquetes ---
\usepackage[utf8]{inputenc}
\usepackage[T1]{fontenc}
\usepackage{graphicx}
\usepackage{geometry}
\usepackage{amsmath,amssymb}
\usepackage{listings}
\usepackage{xcolor}
\usepackage{tcolorbox}
\usepackage{booktabs}
\usepackage{caption}
\usepackage{hyperref}
\usepackage{makecell}
\usepackage{enumitem}

% --- Configuración de página ---
\geometry{a4paper, margin=1in}

% --- Definiciones de colores para código ---
\definecolor{codegreen}{rgb}{0,0.6,0}
\definecolor{codegray}{rgb}{0.5,0.5,0.5}
\definecolor{codepurple}{rgb}{0.58,0,0.82}
\definecolor{backcolour}{rgb}{0.95,0.95,0.92}

% --- Estilo para código Python ---
\lstdefinestyle{pythonstyle}{
    backgroundcolor=\color{backcolour},
    commentstyle=\color{codegreen},
    keywordstyle=\color{magenta},
    numberstyle=\tiny\color{codegray},
    stringstyle=\color{codepurple},
    basicstyle=\ttfamily\footnotesize,
    breaklines=true,
    captionpos=b,
    keepspaces=true,
    tabsize=2,
    language=Python,
    showstringspaces=false,
    numbers=left,
    numbersep=5pt,
    frame=single,
    rulecolor=\color{black}
}

% --- Estilo para salida de consola o bloques de texto ---
\lstdefinestyle{outputstyle}{
    backgroundcolor=\color{backcolour},
    basicstyle=\ttfamily\footnotesize,
    breaklines=true,
    captionpos=b,
    keepspaces=true,
    numbers=none,
    frame=single,
    rulecolor=\color{black},
    showstringspaces=false
}

\begin{document}

% --- Encabezado principal ---
\begin{center}
    {\LARGE\bfseries UNIVERSIDAD NACIONAL DE SAN ANTONIO ABAD DEL CUSCO}\\[0.3cm]
    {\Large FACULTAD DE INGENIERÍA ELÉCTRICA, ELECTRÓNICA, INFORMÁTICA Y MECÁNICA}\\[0.3cm]
    {\Large ESCUELA PROFESIONAL DE INGENIERÍA INFORMÁTICA Y DE SISTEMAS}\\[1cm]
    \includegraphics[width=0.25\linewidth]{Escudo_UNSAAC.png}\\[1cm]
    {\Large\bfseries CURSO: BIOINFORMÁTICA}\\[0.3cm]
    {\Large\bfseries TRABAJO: LABORATORIO 9 }\\[0.3cm]
    {\Large\bfseries PROFESORA: MARIA DEL PILAR VENEGAS VERGARA}\\[0.3cm]
    {\Large\bfseries ALUMNO: EFRAIN VITORINO MARÍN}\\[0.3cm]
    {\Large\bfseries CÓDIGO: 160337}\\[0.3cm]
    {\Large 2025-I}
\end{center}
\newpage

% --- Contenido principal ---
\section{Actividad 1: Fundamentos Teóricos del Ensamblaje de Genomas}

\begin{enumerate}
    \item \textbf{Ensamblaje de Genoma}
    \begin{itemize}
        \item \textbf{Definición:} \\
        Proceso de reconstruir una secuencia genómica $S$ a partir de un conjunto de lecturas (subcadenas) $R = \{r_1, r_2, \dots, r_n\}$.
        \item \textbf{Objetivo Formal (Problema SCS - Shortest Common Superstring):} \\
        Encontrar la cadena $S$ más corta tal que:
        \[ \forall r_i \in R, r_i \text{ es subcadena de } S. \]
        \item \textbf{Complejidad:} NP-Hard (no existe algoritmo eficiente conocido en el caso general).
    \end{itemize}

    \item \textbf{Camino y Ciclo Euleriano}
    \begin{itemize}
        \item \textbf{Definiciones (Grafo Dirigido $G=(V,E)$):}
        \begin{itemize}
            \item \textbf{Camino Euleriano:} Secuencia de aristas que recorre cada arista exactamente una vez.
            \item \textbf{Ciclo Euleriano:} Camino euleriano que comienza y termina en el mismo vértice.
        \end{itemize}
        \item \textbf{Teorema de Euler (Grafos Dirigidos):}
        \begin{itemize}
            \item \textbf{Ciclo Euleriano:}
            \[ \forall v \in V, \text{grado de entrada}(v) = \text{grado de salida}(v) \text{ y } G \text{ es fuertemente conexo.} \]
            \item \textbf{Camino Euleriano:} \\
            Existen vértices $u,v$ (posiblemente iguales) tales que:
            \begin{itemize}
                \item $\text{grado de salida}(u) = \text{grado de entrada}(u) + 1,$
                \item $\text{grado de entrada}(v) = \text{grado de salida}(v) + 1,$
                \item $\forall w \in V, w \neq u, w \neq v, \text{grado de entrada}(w) = \text{grado de salida}(w).$
            \end{itemize}
            Si $u=v$, todos los vértices tienen grados equilibrados (y el camino es un ciclo).
        \end{itemize}
    \end{itemize}

    \item \textbf{Algoritmos para Ensamblaje De Novo}
    \begin{enumerate}[label=\alph*)]
        \item \textbf{Grafos de De Bruijn}
        \begin{itemize}
            \item \textbf{Nodos:} $(k-1)$-mers (subcadenas de longitud $k-1$).
            \item \textbf{Aristas:} $k$-mers (subcadenas de longitud $k$).
            \item \textbf{Construcción:} Para cada $k$-mer $x_1x_2\dots x_k$:
            \begin{itemize}
                \item Nodo inicial $= x_1x_2\dots x_{k-1}$,
                \item Nodo final $= x_2x_3\dots x_k$.
            \end{itemize}
            \item \textbf{Ensamblaje:} Buscar un camino/ciclo euleriano que recorra todas las aristas.
        \end{itemize}

        \item \textbf{OLC (Overlap-Layout-Consensus)}
        \begin{itemize}
            \item \textbf{Overlap (Solapamiento):}
            \[ ov(r_i, r_j) = \text{máximo prefijo de } r_j \text{ que coincide con un sufijo de } r_i. \]
            \item \textbf{Grafo de Solapamiento:}
            \begin{itemize}
                \item Nodos = lecturas $r_i$.
                \item Aristas = solapamientos $ov(r_i, r_j)$.
            \end{itemize}
            \item \textbf{Objetivo:} Encontrar un camino hamiltoniano (NP-Hard, se usan heurísticas).
        \end{itemize}

        \item \textbf{String Graphs}
        \begin{itemize}
            \item Grafo simplificado del OLC, eliminando redundancias.
            \item \textbf{Objetivo:} Identificar caminos únicos sin colapsar ramas divergentes.
            \item \textbf{Ecuación de Consenso:}
            \[ S = \operatorname*{arg\,min}_{S'} \sum_{r_i \in R} \text{edit\_distance}(S', r_i). \]
        \end{itemize}
    \end{enumerate}
\end{enumerate}

implementacion del codigo 
"""
Implementación completa de un ensamblador de genomas usando:
1. Grafos de De Bruijn
2. Algoritmo para encontrar caminos/ciclos eulerianos
3. Ensamblaje de la secuencia genómica final
"""

from collections import defaultdict, deque

class GenomeAssembler:
    def __init__(self, k):
        """
        Inicializa el ensamblador con el tamaño de k-mers a usar
        
        Args:
            k (int): Tamaño de los k-mers para construir el grafo
        """
        self.k = k  # Tamaño de los k-mers
        self.graph = defaultdict(list)  # Grafo de De Bruijn (diccionario de listas)
        self.in_degree = defaultdict(int)  # Contador de grados de entrada
        self.out_degree = defaultdict(int)  # Contador de grados de salida

    def build_de_bruijn_graph(self, reads):
        """
        Construye el grafo de De Bruijn a partir de las lecturas de ADN
        
        Args:
            reads (list): Lista de cadenas de ADN (lecturas)
        """
        for read in reads:
            # Dividir cada lectura en k-mers y crear aristas en el grafo
            for i in range(len(read) - self.k + 1):
                kmer = read[i:i+self.k]
                left = kmer[:-1]  # (k-1)-mer prefijo
                right = kmer[1:]   # (k-1)-mer sufijo
                
                # Añadir arista al grafo
                self.graph[left].append(right)
                # Actualizar grados
                self.out_degree[left] += 1
                self.in_degree[right] += 1

    def find_eulerian_path_or_cycle(self):
        """
        Encuentra un camino o ciclo euleriano en el grafo usando el algoritmo de Hierholzer
        
        Returns:
            list: Lista de nodos en el camino/ciclo euleriano
        """
        # Paso 1: Encontrar nodos inicial y final (para camino euleriano)
        start_node = None
        end_node = None
        
        # Obtener todos los nodos del grafo
        all_nodes = set(self.graph.keys()).union(
            set(x for neighbors in self.graph.values() for x in neighbors))
        
        # Buscar nodos con grados desbalanceados (para camino euleriano)
        for node in all_nodes:
            if self.out_degree.get(node, 0) - self.in_degree.get(node, 0) == 1:
                start_node = node
            elif self.in_degree.get(node, 0) - self.out_degree.get(node, 0) == 1:
                end_node = node
        
        # Si no hay nodos desbalanceados, es un ciclo euleriano
        if start_node is None:
            start_node = next(iter(self.graph.keys())) if self.graph else None
        
        # Paso 2: Algoritmo de Hierholzer para encontrar el camino/ciclo
        stack = []
        path = []
        current_path = [start_node]
        
        while current_path:
            current_node = current_path[-1]
            if self.graph[current_node]:
                # Hay aristas salientes sin visitar
                next_node = self.graph[current_node].pop()
                current_path.append(next_node)
            else:
                # No hay más aristas salientes, añadir al camino
                path.append(current_path.pop())
        
        # Invertir para obtener el orden correcto
        path = path[::-1]
        return path

    def assemble_genome(self, path):
        """
        Ensambla la secuencia genómica a partir del camino euleriano
        
        Args:
            path (list): Camino euleriano (lista de nodos)
            
        Returns:
            str: Secuencia de ADN ensamblada
        """
        if not path:
            return ""
        
        # El genoma comienza con el primer nodo
        genome = path[0]
        
        # Para cada nodo subsiguiente, añadir el último carácter
        for node in path[1:]:
            genome += node[-1]
            
        return genome

    def visualize_graph(self):
        """
        Muestra una representación visual del grafo de De Bruijn
        """
        print("\nGrafo de De Bruijn:")
        print("Formato: Nodo -> Vecino1, Vecino2, ...")
        for node, neighbors in sorted(self.graph.items()):
            print(f"{node} -> {', '.join(neighbors)}")


def run_assembly_case(L, S, K, case_num):
    """
    Ejecuta un caso completo de ensamblaje y muestra los resultados
    
    Args:
        L (list): Lista de lecturas de ADN
        S (int): Tamaño de solapamiento (no usado directamente en esta implementación)
        K (int): Tamaño de k-mers a usar
        case_num (int): Número de caso para mostrar en los resultados
    """
    print(f"\n{'='*50}")
    print(f"Caso de Prueba {case_num}")
    print(f"Lecturas: {L}")
    print(f"K (tamaño de k-mers): {K}")
    print(f"S (solapamiento): {S}")
    print(f"{'='*50}")
    
    # 1. Construir el grafo de De Bruijn
    assembler = GenomeAssembler(K)
    assembler.build_de_bruijn_graph(L)
    
    # 2. Mostrar el grafo
    assembler.visualize_graph()
    
    # 3. Encontrar camino/ciclo euleriano
    path = assembler.find_eulerian_path_or_cycle()
    
    # Determinar si es camino o ciclo
    if path and path[0] == path[-1]:
        print("\nTipo: CICLO euleriano")
    else:
        print("\nTipo: CAMINO euleriano")
    
    print("\nCamino/Ciclo Euleriano encontrado:")
    print(" -> ".join(path))
    
    # 4. Ensamblar el genoma
    genome = assembler.assemble_genome(path)
    print("\nGenoma ensamblado final:")
    print(genome)
    
    return genome


# =============================================================================
# Ejecución de los casos de prueba solicitados
# =============================================================================

if __name__ == "__main__":
    print("="*70)
    print("APLICACIÓN PARA ENSAMBLAJE DE GENOMAS")
    print("="*70)
    print("Este programa implementa:")
    print("- Construcción de grafos de De Bruijn")
    print("- Búsqueda de caminos/ciclos eulerianos")
    print("- Ensamblaje de secuencias genómicas")
    print("="*70)
    
    # Caso 1
    L1 = ['TAG', 'CTG', 'GCT', 'ACT', 'CTA', 'TGC']
    S1 = 1
    K1 = 3
    genome1 = run_assembly_case(L1, S1, K1, 1)
    
    # Caso 2
    L2 = ['TAC', 'CTA', 'TAG', 'ATA', 'CAT', 'ACA', 'CCT', 'GAC', 'CCC', 'ACC']
    S2 = 1
    K2 = 3
    genome2 = run_assembly_case(L2, S2, K2, 2)
    
    # Caso 3
    L3 = ['AACGTTTC', 'TTCCAGTC', 'GTCCAAAT', 'AATAGCTA', 'CTAGGCAT']
    S3 = 3
    K3 = 4
    genome3 = run_assembly_case(L3, S3, K3, 3)
    
    print("\n" + "="*70)
    print("Resumen de resultados:")
    print(f"Caso 1: Genoma ensamblado = {genome1}")
    print(f"Caso 2: Genoma ensamblado = {genome2}")
    print(f"Caso 3: Genoma ensamblado = {genome3}")
    print("="*70)

    resultados de la implementacion 
    
\end{document}
