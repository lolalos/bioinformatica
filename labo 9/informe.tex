\documentclass[fleqn,10pt]{article}

% --- Preámbulo: Paquetes ---
\usepackage[utf8]{inputenc}
\usepackage[T1]{fontenc}
\usepackage{graphicx}
\usepackage{geometry}
\usepackage{amsmath,amssymb}
\usepackage{listings}
\usepackage{xcolor}
\usepackage{tcolorbox}
\usepackage{booktabs}
\usepackage{caption}
\usepackage{hyperref}
\usepackage{makecell}
\usepackage{enumitem}

% --- Configuración de página ---
\geometry{a4paper, margin=1in}

% --- Definiciones de colores para código ---
\definecolor{codegreen}{rgb}{0,0.6,0}
\definecolor{codegray}{rgb}{0.5,0.5,0.5}
\definecolor{codepurple}{rgb}{0.58,0,0.82}
\definecolor{backcolour}{rgb}{0.95,0.95,0.92}

% --- Estilo para código Python ---
\lstdefinestyle{pythonstyle}{
    backgroundcolor=\color{backcolour},
    commentstyle=\color{codegreen},
    keywordstyle=\color{magenta},
    numberstyle=\tiny\color{codegray},
    stringstyle=\color{codepurple},
    basicstyle=\ttfamily\footnotesize,
    breaklines=true,
    captionpos=b,
    keepspaces=true,
    tabsize=2,
    language=Python,
    showstringspaces=false,
    numbers=left,
    numbersep=5pt,
    frame=single,
    rulecolor=\color{black}
}

% --- Estilo para salida de consola o bloques de texto ---
\lstdefinestyle{outputstyle}{
    backgroundcolor=\color{backcolour},
    basicstyle=\ttfamily\footnotesize,
    breaklines=true,
    captionpos=b,
    keepspaces=true,
    numbers=none,
    frame=single,
    rulecolor=\color{black},
    showstringspaces=false
}

\begin{document}

% --- Encabezado principal ---
\begin{center}
    {\LARGE\bfseries UNIVERSIDAD NACIONAL DE SAN ANTONIO ABAD DEL CUSCO}\\[0.3cm]
    {\Large FACULTAD DE INGENIERÍA ELÉCTRICA, ELECTRÓNICA, INFORMÁTICA Y MECÁNICA}\\[0.3cm]
    {\Large ESCUELA PROFESIONAL DE INGENIERÍA INFORMÁTICA Y DE SISTEMAS}\\[1cm]
    \includegraphics[width=0.25\linewidth]{Escudo_UNSAAC.png}\\[1cm]
    {\Large\bfseries CURSO: BIOINFORMÁTICA}\\[0.3cm]
    {\Large\bfseries TRABAJO: LABORATORIO 9 }\\[0.3cm]
    {\Large\bfseries PROFESORA: MARIA DEL PILAR VENEGAS VERGARA}\\[0.3cm]
    {\Large\bfseries ALUMNO: EFRAIN VITORINO MARÍN}\\[0.3cm]
    {\Large\bfseries CÓDIGO: 160337}\\[0.3cm]
    {\Large 2025-I}
\end{center}
\newpage

% --- Contenido principal ---
\section{Actividad 1: Fundamentos Teóricos del Ensamblaje de Genomas}

\begin{enumerate}
    \item \textbf{Ensamblaje de Genoma}
    \begin{itemize}
        \item \textbf{Definición:} \\
        Proceso de reconstruir una secuencia genómica $S$ a partir de un conjunto de lecturas (subcadenas) $R = \{r_1, r_2, \dots, r_n\}$.
        \item \textbf{Objetivo Formal (Problema SCS - Shortest Common Superstring):} \\
        Encontrar la cadena $S$ más corta tal que:
        \[ \forall r_i \in R, r_i \text{ es subcadena de } S. \]
        \item \textbf{Complejidad:} NP-Hard (no existe algoritmo eficiente conocido en el caso general).
    \end{itemize}

    \item \textbf{Camino y Ciclo Euleriano}
    \begin{itemize}
        \item \textbf{Definiciones (Grafo Dirigido $G=(V,E)$):}
        \begin{itemize}
            \item \textbf{Camino Euleriano:} Secuencia de aristas que recorre cada arista exactamente una vez.
            \item \textbf{Ciclo Euleriano:} Camino euleriano que comienza y termina en el mismo vértice.
        \end{itemize}
        \item \textbf{Teorema de Euler (Grafos Dirigidos):}
        \begin{itemize}
            \item \textbf{Ciclo Euleriano:}
            \[ \forall v \in V, \text{grado de entrada}(v) = \text{grado de salida}(v) \text{ y } G \text{ es fuertemente conexo.} \]
            \item \textbf{Camino Euleriano:} \\
            Existen vértices $u,v$ (posiblemente iguales) tales que:
            \begin{itemize}
                \item $\text{grado de salida}(u) = \text{grado de entrada}(u) + 1,$
                \item $\text{grado de entrada}(v) = \text{grado de salida}(v) + 1,$
                \item $\forall w \in V, w \neq u, w \neq v, \text{grado de entrada}(w) = \text{grado de salida}(w).$
            \end{itemize}
            Si $u=v$, todos los vértices tienen grados equilibrados (y el camino es un ciclo).
        \end{itemize}
    \end{itemize}

    \item \textbf{Algoritmos para Ensamblaje De Novo}
    \begin{enumerate}[label=\alph*)]
        \item \textbf{Grafos de De Bruijn}
        \begin{itemize}
            \item \textbf{Nodos:} $(k-1)$-mers (subcadenas de longitud $k-1$).
            \item \textbf{Aristas:} $k$-mers (subcadenas de longitud $k$).
            \item \textbf{Construcción:} Para cada $k$-mer $x_1x_2\dots x_k$:
            \begin{itemize}
                \item Nodo inicial $= x_1x_2\dots x_{k-1}$,
                \item Nodo final $= x_2x_3\dots x_k$.
            \end{itemize}
            \item \textbf{Ensamblaje:} Buscar un camino/ciclo euleriano que recorra todas las aristas.
        \end{itemize}

        \item \textbf{OLC (Overlap-Layout-Consensus)}
        \begin{itemize}
            \item \textbf{Overlap (Solapamiento):}
            \[ ov(r_i, r_j) = \text{máximo prefijo de } r_j \text{ que coincide con un sufijo de } r_i. \]
            \item \textbf{Grafo de Solapamiento:}
            \begin{itemize}
                \item Nodos = lecturas $r_i$.
                \item Aristas = solapamientos $ov(r_i, r_j)$.
            \end{itemize}
            \item \textbf{Objetivo:} Encontrar un camino hamiltoniano (NP-Hard, se usan heurísticas).
        \end{itemize}

        \item \textbf{String Graphs}
        \begin{itemize}
            \item Grafo simplificado del OLC, eliminando redundancias.
            \item \textbf{Objetivo:} Identificar caminos únicos sin colapsar ramas divergentes.
            \item \textbf{Ecuación de Consenso:}
            \[ S = \operatorname*{arg\,min}_{S'} \sum_{r_i \in R} \text{edit\_distance}(S', r_i). \]
        \end{itemize}
    \end{enumerate}
\end{enumerate}

\end{document}
