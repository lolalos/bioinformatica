\documentclass[fleqn,10pt]{article}
\usepackage[utf8]{inputenc}
\usepackage[T1]{fontenc}
\usepackage{graphicx}
\usepackage{geometry}
\usepackage{amsmath,amssymb}
\usepackage{listings}
\usepackage{xcolor}
\usepackage{tcolorbox}
\usepackage{booktabs}
\usepackage{caption}
\usepackage{amssymb}

\geometry{a4paper, margin=1in}

% Estilo para código Python
\definecolor{codegreen}{rgb}{0,0.6,0}
\definecolor{codegray}{rgb}{0.5,0.5,0.5}
\definecolor{codepurple}{rgb}{0.58,0,0.82}
\definecolor{backcolour}{rgb}{0.95,0.95,0.92}

\lstdefinestyle{pythonstyle}{
    backgroundcolor=\color{backcolour},
    commentstyle=\color{codegreen},
    keywordstyle=\color{magenta},
    numberstyle=\tiny\color{codegray},
    stringstyle=\color{codepurple},
    basicstyle=\ttfamily\footnotesize,
    breaklines=true,
    captionpos=b,
    keepspaces=true,
    tabsize=2,
    language=Python,
    showstringspaces=false,
    numbers=left,
    numbersep=5pt,
    frame=single,
    rulecolor=\color{black}
}

\begin{document}

%---------------------- portada ----------------------
\begin{titlepage}
    \centering
    \vspace*{1cm}
    {\LARGE\bfseries UNIVERSIDAD NACIONAL DE SAN ANTONIO ABAD DEL CUSCO\par}
    \vspace{0.5cm}
    {\Large FACULTAD DE INGENIERÍA ELÉCTRICA, ELECTRÓNICA, INFORMÁTICA Y MECÁNICA\par}
    \vspace{0.5cm}
    {\Large ESCUELA PROFESIONAL DE INGENIERÍA INFORMÁTICA Y DE SISTEMAS\par}
    \vfill
    \includegraphics[width=0.25\linewidth]{Escudo_UNSAAC.png}\par
    \vfill
    {\Large\bfseries CURSO: BIOINFORMÁTICA\par}
    \vspace{0.3cm}
    {\Large\bfseries TRABAJO: BÚSQUEDA DE PATRONES\par}
    \vspace{0.3cm}
    {\Large\bfseries PROFESORA: MARIA DEL PILAR VENEGAS VERGARA\par}
    \vspace{1cm}
    {\Large\bfseries ALUMNO: EFRAIN VITORINO MARÍN\par}
    {\Large\bfseries CÓDIGO: 160337\par}
    \vfill
    {\Large 2025‑I\par}
\end{titlepage}

\setcounter{page}{1}
\pagestyle{plain}
\tableofcontents
\newpage

%---------------------- contenido ----------------------

\section{¿Qué es alineamiento de secuencias?}
\textbf{Definición:} \\
El alineamiento de secuencias es una técnica utilizada para comparar dos o más secuencias biológicas (ADN, ARN o proteínas) con el fin de identificar regiones de similitud que pueden indicar relaciones funcionales, estructurales o evolutivas.

\textbf{Aplicaciones:}
\begin{itemize}
    \item Comparación de genes entre especies.
    \item Identificación de mutaciones.
    \item Predicción de estructuras de proteínas.
    \item Análisis filogenético.
\end{itemize}

\section{¿Qué es alineamiento global?}
\textbf{Definición:} \\
El alineamiento global busca alinear completamente dos secuencias desde el inicio hasta el final, optimizando la coincidencia total a lo largo de toda su longitud.

\textbf{Aplicación principal:} \\
Comparación de secuencias de longitud similar que se espera estén altamente relacionadas (por ejemplo, dos genes homólogos).

\textbf{Algoritmo común:} Needleman-Wunsch.

\textbf{Fórmula básica (recursiva):}

Sea $F(i,j)$ la puntuación óptima para alinear los primeros $i$ caracteres de la secuencia A y los primeros $j$ de la secuencia B:

\[
F(i,j) = \max \begin{cases}
F(i-1, j-1) + s(a_i, b_j) & \text{(emparejamiento)} \\
F(i-1, j) + d & \text{(inserción)} \\
F(i, j-1) + d & \text{(eliminación)}
\end{cases}
\]

Donde:
\begin{itemize}
    \item $s(a_i, b_j)$ es la puntuación por coincidencia o desajuste.
    \item $d$ es la penalización por gap (inserción o eliminación).
\end{itemize}

\section{¿Qué es alineamiento local?}
\textbf{Definición:} \\
El alineamiento local encuentra la región de mayor similitud entre dos secuencias, sin requerir que el alineamiento cubra toda la longitud de ambas.

\textbf{Aplicación principal:} \\
Detección de dominios conservados entre proteínas o regiones similares en genes que no son totalmente homólogos.

\textbf{Algoritmo común:} Smith-Waterman.

\textbf{Fórmula básica (recursiva):}

\[
F(i,j) = \max \begin{cases}
0 \\
F(i-1, j-1) + s(a_i, b_j) \\
F(i-1, j) + d \\
F(i, j-1) + d
\end{cases}
\]

El valor $0$ asegura que se puedan descartar partes sin coincidencia.

\section{¿Cuáles son las características del algoritmo del alineamiento Needleman-Wunsch?}
\begin{itemize}
    \item Es un algoritmo de alineamiento global.
    \item Utiliza programación dinámica.
    \item Rellena una matriz de puntuación para encontrar el alineamiento óptimo.
    \item Tiene una complejidad de tiempo y espacio de $O(n \times m)$, donde $n$ y $m$ son las longitudes de las secuencias.
    \item Se usa cuando las secuencias tienen longitudes similares y se desea alinear toda la secuencia.
    \item Fórmula: (ya mencionada arriba).
\end{itemize}

\section{¿Cuáles son las características del algoritmo del alineamiento Smith-Waterman?}
\begin{itemize}
    \item Es un algoritmo de alineamiento local.
    \item También utiliza programación dinámica.
    \item Se enfoca en encontrar subsecuencias con alta similitud.
    \item Permite descartar regiones no similares mediante el uso del valor $0$ en la matriz.
    \item Ideal para encontrar dominios conservados.
    \item Complejidad: $O(n \times m)$, pero puede ser optimizado con heurísticas o técnicas como BLAST.
    \item Fórmula: (ya mencionada arriba).
\end{itemize}

\end{document}
