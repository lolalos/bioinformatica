\documentclass[fleqn,10pt]{article}

% --- Preámbulo: Paquetes ---
\usepackage[utf8]{inputenc}
\usepackage[T1]{fontenc}
\usepackage{graphicx}
\usepackage{geometry}
\usepackage{amsmath,amssymb}
\usepackage{listings}
\usepackage{xcolor}
\usepackage{tcolorbox}
\usepackage{booktabs}
\usepackage{caption}
\usepackage{hyperref}
\usepackage{makecell}
\usepackage{enumitem}

% --- Configuración de página ---
\geometry{a4paper, margin=1in}

% --- Definiciones de colores para código ---
\definecolor{codegreen}{rgb}{0,0.6,0}
\definecolor{codegray}{rgb}{0.5,0.5,0.5}
\definecolor{codepurple}{rgb}{0.58,0,0.82}
\definecolor{backcolour}{rgb}{0.95,0.95,0.92}

% --- Estilo para código Python ---
\lstdefinestyle{pythonstyle}{
    backgroundcolor=\color{backcolour},
    commentstyle=\color{codegreen},
    keywordstyle=\color{magenta},
    numberstyle=\tiny\color{codegray},
    stringstyle=\color{codepurple},
    basicstyle=\ttfamily\footnotesize,
    breaklines=true,
    captionpos=b,
    keepspaces=true,
    tabsize=2,
    language=Python,
    showstringspaces=false,
    numbers=left,
    numbersep=5pt,
    frame=single,
    rulecolor=\color{black}
}

% --- Estilo para salida de consola o bloques de texto ---
\lstdefinestyle{outputstyle}{
    backgroundcolor=\color{backcolour},
    basicstyle=\ttfamily\footnotesize,
    breaklines=true,
    captionpos=b,
    keepspaces=true,
    numbers=none,
    frame=single,
    rulecolor=\color{black},
    showstringspaces=false
}

\begin{document}

% --- Encabezado principal ---
\begin{center}
    {\LARGE\bfseries UNIVERSIDAD NACIONAL DE SAN ANTONIO ABAD DEL CUSCO}\\[0.3cm]
    {\Large FACULTAD DE INGENIERÍA ELÉCTRICA, ELECTRÓNICA, INFORMÁTICA Y MECÁNICA}\\[0.3cm]
    {\Large ESCUELA PROFESIONAL DE INGENIERÍA INFORMÁTICA Y DE SISTEMAS}\\[1cm]
    \includegraphics[width=0.25\linewidth]{Escudo_UNSAAC.png}\\[1cm]
    {\Large\bfseries CURSO: BIOINFORMÁTICA}\\[0.3cm]
    {\Large\bfseries TRABAJO: LABORATORIO 8 }\\[0.3cm]
    {\Large\bfseries PROFESORA: MARIA DEL PILAR VENEGAS VERGARA}\\[0.3cm]
    {\Large\bfseries ALUMNO: EFRAIN VITORINO MARÍN}\\[0.3cm]
    {\Large\bfseries CÓDIGO: 160337}\\[0.3cm]
    {\Large 2025-I}
\end{center}
\newpage

% --- Contenido principal ---
\section{Actividad 1: Responder a las siguientes interrogantes}

\subsection{1. ¿Qué es alineamiento múltiple?}

El \textbf{alineamiento múltiple} es una extensión del alineamiento por pares que permite comparar simultáneamente tres o más secuencias de ADN, ARN o proteínas para identificar regiones de similitud que pueden indicar relaciones funcionales, estructurales o evolutivas.

\begin{tcolorbox}[breakable, colback=blue!5!white,colframe=blue!75!black,title=Definición Formal]
Sea $S = \{s_1, s_2, \ldots, s_k\}$ un conjunto de $k$ secuencias sobre un alfabeto $\Sigma$. Un alineamiento múltiple $A$ es una matriz donde:
\begin{itemize}
    \item Cada fila $i$ representa la secuencia $s_i$ con posibles gaps (-)
    \item Todas las filas tienen la misma longitud $L$
    \item Al remover los gaps de la fila $i$, se obtiene la secuencia original $s_i$
\end{itemize}
\end{tcolorbox}

\textbf{Características principales:}
\begin{enumerate}[label=\textbf{\alph*)}]
    \item \textbf{Conservación evolutiva}: Identifica regiones conservadas entre especies
    \item \textbf{Análisis funcional}: Revela sitios activos y dominios funcionales
    \item \textbf{Predicción estructural}: Ayuda a predecir estructuras secundarias y terciarias
\end{enumerate}

\subsection{2. ¿Qué diferencias tienen los algoritmos de alineamiento múltiple, frente a alineamiento local y global?}

\begin{table}[htbp]
\centering
\caption{Comparación entre tipos de alineamiento}
\begin{tabular}{@{}lccc@{}}
\toprule
\textbf{Característica} & \textbf{Global} & \textbf{Local} & \textbf{Múltiple} \\
\midrule
\makecell{Número de \\ secuencias} & 2 & 2 & $\geq 3$ \\
\makecell{Cobertura de \\ la secuencia} & Completa & Parcial & Variable \\
\makecell{Complejidad \\ temporal} & $O(mn)$ & $O(mn)$ & $O(L^k)$ \\
\makecell{Algoritmo \\ principal} & \makecell{Needleman-\\Wunsch} & \makecell{Smith-\\Waterman} & \makecell{ClustalW, MUSCLE,\\T-Coffee} \\
\makecell{Función \\ objetivo} & \makecell{Maximizar similitud\\global} & \makecell{Encontrar regiones\\similares} & \makecell{Optimizar suma\\de pares} \\
\bottomrule
\end{tabular}
\end{table}

\begin{tcolorbox}[breakable, colback=green!5!white,colframe=green!75!black,title=Teorema de Complejidad]
\textbf{Teorema}: El problema de alineamiento múltiple óptimo es NP-completo.

\textbf{Demostración}: Se reduce del problema de la subsecuencia común más larga (LCS) para múltiples secuencias, que es conocido como NP-completo.

\textbf{Implicación}: Para $k$ secuencias de longitud promedio $n$, la complejidad es $O(n^k)$, lo que hace inviable la solución exacta para grandes valores de $k$.
\end{tcolorbox}

\textbf{Diferencias fundamentales:}

\begin{enumerate}[label=\textbf{\Roman*.}]
    \item \textbf{Dimensionalidad del problema}:
    \begin{itemize}
        \item Alineamiento por pares: Matriz 2D
        \item Alineamiento múltiple: Hipermatriz $k$-dimensional
    \end{itemize}
    
    \item \textbf{Estrategias algorítmicas}:
    \begin{itemize}
        \item \textbf{Progresivas}: Construyen el alineamiento paso a paso (ClustalW)
        \item \textbf{Iterativas}: Refinan alineamientos iniciales (MUSCLE)
        \item \textbf{Consistencia}: Maximizan la consistencia entre alineamientos por pares (T-Coffee)
    \end{itemize}
    
    \item \textbf{Función de puntuación}:
    \begin{align}
        \text{Score}_{\text{múltiple}} &= \sum_{i<j} \text{Score}(s_i, s_j) \quad \text{(Suma de pares)} \\
        \text{Score}_{\text{global}} &= \max_{A} \sum_{i=1}^{L} \sigma(A[i,1], A[i,2]) \\
        \text{Score}_{\text{local}} &= \max_{A,i,j} \sum_{k=i}^{j} \sigma(A[k,1], A[k,2])
    \end{align}
\end{enumerate}

\begin{tcolorbox}[breakable, colback=red!5!white,colframe=red!75!black,title=Limitaciones Computacionales]
\textbf{Problema}: La programación dinámica exacta para $k$ secuencias requiere:
\begin{itemize}
    \item Espacio: $O(n^k)$
    \item Tiempo: $O(kn^k)$
\end{itemize}

\textbf{Solución}: Uso de heurísticas y aproximaciones que sacrifican optimalidad por eficiencia computacional.
\end{tcolorbox}
\newpage

\section{Actividad 2: Utilizar el algoritmo Blast de NCBI para alinear las secuencias de la tabla 1}

\begin{table}[htbp]
\centering
\caption{Secuencias para alineamiento múltiple}
\begin{tabular}{|p{2cm}|p{12cm}|}
\hline
\textbf{Secuencia} & \textbf{Secuencia completa} \\
\hline
Secuencia 1 & \texttt{MMALGRAFAIVFCLIQAVSGESGNAQDGDLEDADADDHSFWCHSQLEVDGSQHLLTCAFNDSDINTANLEFQI CGALLRVKCLTLNKLQDIYFIKTSEFLLIGSSNICVKLGQKNLTCKNMAINTIVKAEAPSDLKVVYRKEANDF LVTFNAPHLKKKYLKKVKHDVAYRPARGESNWTHVSLFHTRTTIPQRKLRPKAMYEIKVRSIPHNDYFKGFWS EWSPSSTFETPEPKNQGGWDPVLPSVTILSLFSVFLLVILAHLVLWKRIKPVVWPSLPDHKKTLEQLCHKPKT SLNVSFNPESFLDCQIHEVKGVEARDEVESFLPNDLPAQPEELETNIPQGHRAAVHSANRSPETSVSPPLNKL RESPLRCLATCNAPPLLSSRSPDYRDGDRNRPPVYQDLLPNSGNTNVPVPVPQPLPFQSGILIPVSQRQPIST SSVLNQEEAYVTMSSFYQNK} \\
\hline
Secuencia 2 & \texttt{NRGETGAPAGPRGPAGPAGSSGKDGVGGLPGPIGPPSPRGRTGDIGPAGPPGTPGPPGPPGPPGGGFDFSFVA QPSQEKAPDPFRHYRADDANVARDRDLEVDTTLKSLSQQKDLAIENIRSPEGTKKDPARSCRDLKMCHPEWKS GEYFVDPNQGCDEDAVKVYCNMETGETCVYPTQANIPQKNWYTSKNAKDKKHVWFGETMSDGFQFEYGGEGSD AADVNIQLTFLRLMATEASQNITYHCKNSIAYMDQQAGNLKKALLLQGSNEIEIRAEGNSRFTYSEETEDGYT RHTGAWGKTVIDADYKTTKTSRLPIIDIAPMDVGAPDQEFGIDVGP} \\
\hline
Secuencia 3 & \texttt{MSFSRRPKITKSDIVDTVYFQISLNIRNNNLKLEKKKIRLVIDAFFEELKGNLALNNVIEFRSFGTEVRKRK GRLNPRSEYKVLHDHVAYYHTYQGFPSHSCHIPKDLALFTFYEIWVEATNRRGSARSDVLTLDEVDTVTTDPPP EVHVSRVGGLEDQLSVRWVSPPALKLKERVWGIKG} \\
\hline
Secuencia 4 & \texttt{TGGGATGATTCCACACCCGCGCCCGGCACCCGCGTCCGCGCCGTAGCCATCAACAAGCAGTCACAGCACATGA CGGAGGTTGTGAGGCGCTGCCTCCACGATGAGCGCTACTCAGATAGCGATA} \\
\hline
Secuencia 5 & \texttt{ACTATAAAGGCGTCAAGCCGTGTTCTAGATAATAATAAGTATTGGGCAACTTATTAGTCTCCGGTCCAACAAC CTGAACGGATTTGATGAAATGGGC} \\
\hline
Secuencia 6 & \texttt{ATATTGGTGTGTGAGGCGTTATAATTCCAAGAAGCAAGTGAACTTTGATAGAACAGGTCTTCGGCTTCGTGGT TAAACTTGTCCAAATGTGAGGCGGCCTGTTCCTCAATGGTGGACTGAGCAGCAGTTACAGCAACAAGGCTGAG AAGGAGCCAGGAAGAGCTTGACATCGTCGCCTCCACAGCCAAGATCACATCCACTGAATGACTTTCCCTAGAC TAAAACCTCCTCATGAGATTTTCTCTCTTATCAGCCTTTGAACTTGGGTTGGGCGCTGAGCAGGAAAGACCAA AAAAAGAAAAAGAAGAAGAACACAGTAAACAATCTGCTGAGCCAATATAAAGTTCATCCTGGAGAGGACAGAT ATGTAACAGATTTTAGAATAATTTTTTAAAGTGAATCAAATAAGAATACGTTATTCTTTAATCCTAGAGAACC TTATCACCTCCGGTCAAATCTCAGGTATCTTGGGGCCCGAGGGCCCAGTATGTCCACGATGCATACCTGCAGA TAAAGATCGCGTCTTGGGTGAGGGCTCCGCGTTATCAATTGGGTCCCCGAACTGGGAAGACTGAAATGCTAGT TTGCGAGTATATAAGAAGACCTCTATAGTGCGAGTATAAGATCATCGAAGAAGGTCGGCGGCTTGTCCGTTTA CTCACTGCTCTTGTGACATAGTAACAACAAGTAACCTCGCCTTAATTGACTGAAGGCATTCCTCGTGCAGTGT GAGGCG} \\
\hline
\end{tabular}
\end{table}


\subsection*{Resultados de BLAST para las secuencias de la tabla}

Las secuencias 1 y 2 son proteínas largas, mientras que la 3 es más corta.

A continuación se muestra un ejemplo de resultado obtenido al alinear las secuencias 1 y 2 utilizando el algoritmo de Needleman-Wunsch (BLAST NCBI):

\begin{tcolorbox}[breakable, colback=gray!5!white, colframe=gray!75!black, title=Resultado de alineamiento (BLAST NCBI)]
\textbf{Título profesional:} Secuencia de proteínas

\textbf{ID de consulta:} lcl|Consulta\_2676595 (aminoácido)\\
\textbf{Descripción de la consulta:} producto proteico sin nombre\\
\textbf{Longitud de la consulta:} 458

\textbf{ID de sujeto:} lcl|Consulta\_2676597 (aminoácido)\\
\textbf{Descripción del tema:} Ninguno\\
\textbf{Longitud del tema:} 338

\textit{Nota: La búsqueda expira el 29/05 a las 06:37.}
\end{tcolorbox}
% \newpage % Removed this newpage to allow more flexible page breaking

\begin{lstlisting}[style=outputstyle,caption={Resultado detallado del alineamiento entre Secuencia 1 y Secuencia 2}]
resultado de alineamiento 
ID de secuencia: Consulta_2676597 Longitud: 338 Número de coincidencias: 1
Rango 1: 1 a 338 GráficosPróximo partidoPartido anterior
Estadísticas de alineación para el partido n.º 1
Puntuación NW	Identidades	Aspectos positivos	Brechas
-213	72/499(14%)	115/499(23%)	202/499(40%)
Consulta 1 MMALGRAFAI---------------VFCLIQAVSGESGNAQDGDLEDADAD--------- 36
                  AAVL + S + GD + A            
Sbjct 1 NRGETGAPAGPRGPAGPAGSSGKDGVGGLPGPIGPPSPRGRTGDIGPAGPPGTPGPPGPP 60

Consulta 37 -------DHSFWCHSQLEV--DGSQHLLTCAFNDSDINTA---NLEFQICGALLRVKCLT 84
                   D SF ED + H + ADN + LE T
Sbjct 61 GPPGGGGFDFSFVAQPSQEKAPDPFRH-----YRADDANVARDRDLEVDT----------T 105

Consulta 85 LNKLQDIYFIKTSEFLLIGSSNICVKLGQKNLTCKNMAINTIVK---AEAPSDLKVVYRK 141
            LL QK+L +N+ KA + DLK+ + +
Objeto 106 LKSLSQ----------------------QKDLAIENIRSPEGTKKDPARSCRDLKMCHPE 143

Consulta 142 -EANDFLVTFNAPHLKKKYLKKVKHDVAYRPARGESNWTHVSLFHTRTTIPQRKLRPKAM 200
             ++ ++ VN + D E+ TV + T+ IPQ+       
Objeto 144 WKSGEYFVDPN---------QGCDEDAVKVYCNMETGETCV--YPTQANIPQKNWYTSKN 192

Consulta 201 YEIKVRSIPHNDYFKGFWSEWSPSSTFETPEPKNQGGWDPVLPSVTILSLFSVFLLVILA 260
             + KKW + S F+ V +TLL +        
Objeto 193 AKDK----------KHVWFGETMSDGFQFEYGGEGSDAADVNIQLTFLRLMAT------- 235

Consulta 261 HVLWKKRIKPVVWPSLPDHKKTLEQLCHKPKTSLNVSFNPESFLDCQIHEVKGVEARDEV 320
                                          + S N+++ H + D+
Sbjct 236 ------------------------------FACILIDAD----------HCKNSIAYMDQQ 255

Consulta 321 ESFLPNDLPAQ-PEELETNIPQGHRAAVHSANRSPETSVSPPLNKLRESPLRCLATCNAP 379
               LLQ E+ERT + K               
Sbjct 256 AGNLKKALLLQGSNEIEIRAEGNSRFTYSEETEDGYTRHTGAWGK--------------- 300

Consulta 380 PLLSSRSPDYRDGDRNRPPVYQDLLPNSGNTNVPVPVPQPLPFQSGILIPVSQRQPISTS 439
               + DY+ +R P+ D+ P + VP + GI +           
Sbjct 301 ---TVIDADYKTTKTSRLPII-DIAP------MDVGAPDQ---EFGIDV----------- 336

Consulta 440 SVLNQEEAYVTMSSFYQNK 458
                               
Sbjct 337 -----------------GP 338
\end{lstlisting}

Gráficos de: \textbf{unnamed protein product} lcl|Query\_2061159 
\begin{figure}[htbp]
    \centering
    \includegraphics[width=1\linewidth]{1 y2.png}
    \caption{Alineamiento entre Secuencia 1 y Secuencia 2}
    \label{fig:aln-1-2}
\end{figure}

\textbf{RESUMEN}
\begin{table}[htbp]
\centering
\begin{tabular}{|l|l|l|l|l|l|l|}
\hline
\textbf{Especie} & \textbf{Región genómica} & \textbf{Función biológica} & \textbf{Score} & \textbf{Relación (BLAST)} & \textbf{Identidad} & \textbf{Brechas} \\
\hline
No especificada & Producto proteico sin nombre & No determinada & -213 & Parecido no significativo & 14\% (Positivos: 23\%) & 40\% \\
\hline
\end{tabular}
\caption{Resumen del alineamiento entre Secuencia 1 y Secuencia 2}
\end{table}
\subsection*{Alineamiento entre Secuencia 3 y Secuencia 4}

Las secuencias 3 y 4 corresponden a diferentes tipos de biomoléculas: la Secuencia 3 es una proteína (aminoácidos) y la Secuencia 4 es ADN (nucleótidos). Por lo tanto, no es posible realizar un alineamiento directo entre ambas usando BLAST estándar, ya que requieren formatos y algoritmos distintos (BLASTp para proteínas, BLASTn para nucleótidos, BLASTx/tBLASTn para traducciones).

\begin{tcolorbox}[breakable, colback=yellow!5!white, colframe=yellow!75!black, title=Nota Importante]
\textbf{Error de BLAST:} \\
\texttt{Error: No es posible alinear directamente una secuencia de proteína con una de nucleótidos.}

Esto indica que BLAST no puede realizar el alineamiento ni como nucleótido ni como proteína, ya que requieren tipos de entrada compatibles (proteína vs proteína o nucleótido vs nucleótido).
\end{tcolorbox}

\begin{figure}[htbp]
    \centering
    \includegraphics[width=0.75\linewidth]{2 y3.jpg}
    \caption{Intento de alineamiento entre Secuencia 3 (proteína) y Secuencia 4 (ADN). No es posible realizar el alineamiento directo debido a la diferencia de tipo de secuencia.}
    \label{fig:aln-3-4}
\end{figure}
\subsection*{Alineamiento entre Secuencia 5 y Secuencia 6}

A continuación se presenta el resultado del alineamiento entre las Secuencias 5 y 6 utilizando el algoritmo de Needleman-Wunsch (BLAST NCBI):

\begin{tcolorbox}[breakable, colback=gray!5!white, colframe=gray!75!black, title=Resultado de alineamiento (BLAST NCBI)]
\textbf{Título del trabajo:} Secuencia de nucleótidos

\textbf{ID de consulta:} lcl|Query\_3909303 (ADN)\\
\textbf{Descripción de la consulta:} Ninguna\\
\textbf{Longitud de la consulta:} 97

\textbf{ID de sujeto:} lcl|Query\_3909305 (ADN)\\
\textbf{Descripción del sujeto:} Ninguna\\
\textbf{Longitud del sujeto:} 736

\textit{Nota: La búsqueda expira el 29/05 a las 07:31.}
\end{tcolorbox}

\begin{lstlisting}[style=outputstyle,caption={Resultado detallado del alineamiento entre Secuencia 5 y Secuencia 6}]
Sequence ID: Query_3909305 Length: 736 Number of Matches: 1
Range 1: 1 to 736 Graphics Next Match Previous Match
Alignment statistics for match #1
NW Score   Identities   Gaps   Strand
-1234      85/736(12%)  639/736(86%)  Plus/Plus

Query  1    AC-----------------TATAA------------------------------------  7
            |                  |||||                                    
Sbjct  1    ATATTGGTGTGTGAGGCGTTATAATTCCAAGAAGCAAGTGAACTTTGATAGAACAGGTCT  60

Query  8    -------------------------------AGGCG---------TCAA-----------  16
                                           |||||         ||||           
Sbjct  61   TCGGCTTCGTGGTTAAACTTGTCCAAATGTGAGGCGGCCTGTTCCTCAATGGTGGACTGA  120

Query  17   ---GCCGT----------------------------------------------------  21
               || ||                                                    
Sbjct  121  GCAGCAGTTACAGCAACAAGGCTGAGAAGGAGCCAGGAAGAGCTTGACATCGTCGCCTCC  180

Query       ------------------------------------------------------------  
                                                                        
Sbjct  181  ACAGCCAAGATCACATCCACTGAATGACTTTCCCTAGACTAAAACCTCCTCATGAGATTT  240

Query       ------------------------------------------------------------  
                                                                        
Sbjct  241  TCTCTCTTATCAGCCTTTGAACTTGGGTTGGGCGCTGAGCAGGAAAGACCAAAAAAAGAA  300

Query  22   -------------------------------------------GTTC-------------  25
                                                       ||||             
Sbjct  301  AAAGAAGAAGAACACAGTAAACAATCTGCTGAGCCAATATAAAGTTCATCCTGGAGAGGA  360

Query  26   ------------------TAGA-TAAT----------------AATAAG-------TATT  43
                              |||| ||||                ||||||       ||||
Sbjct  361  CAGATATGTAACAGATTTTAGAATAATTTTTTAAAGTGAATCAAATAAGAATACGTTATT  420

Query  44   -----------GGGCAACTTATTA------------------------------------  56
                       | | | ||||| |                                    
Sbjct  421  CTTTAATCCTAGAGAACCTTATCACCTCCGGTCAAATCTCAGGTATCTTGGGGCCCGAGG  480

Query  57   -----------------------------------------GTCT------------CCG  63
                                                     ||||            |||
Sbjct  481  GCCCAGTATGTCCACGATGCATACCTGCAGATAAAGATCGCGTCTTGGGTGAGGGCTCCG  540

Query  64   -------------GTCC-------------------------------------------  67
                         ||||                                           
Sbjct  541  CGTTATCAATTGGGTCCCCGAACTGGGAAGACTGAAATGCTAGTTTGCGAGTATATAAGA  600

Query       ------------------------------------------------------------  
                                                                        
Sbjct  601  AGACCTCTATAGTGCGAGTATAAGATCATCGAAGAAGGTCGGCGGCTTGTCCGTTTACTC  660

Query  68   -------------------AACAAC---------------------CTGAACGGATT--T  85
                               ||||||                     ||||| | |||  |
Sbjct  661  ACTGCTCTTGTGACATAGTAACAACAAGTAACCTCGCCTTAATTGACTGAAGGCATTCCT  720

Query  86   GATGAAATG---GGC-  97
              || | ||   ||| 
Sbjct  721  CGTGCAGTGTGAGGCG  736
\end{lstlisting}



\textbf{Datos relevantes de las secuencias alineadas:}
\begin{itemize}
    \item \textbf{Tipo de molécula:} Ambas secuencias son \textbf{ADN} (ácido desoxirribonucleico).
    \item \textbf{Especie:} No especificada en el resultado; normalmente se obtiene de la base de datos NCBI si se consulta con identificadores reales.
    \item \textbf{Explicación de la muestra (NCBI):} Las secuencias corresponden a fragmentos de ADN proporcionados para el ejercicio; en un análisis real, la base de datos NCBI mostraría detalles como fuente, organismo, y anotaciones funcionales.
    \item \textbf{Cromosoma:} No determinado; en NCBI, este dato aparece si la secuencia está mapeada a un cromosoma específico.
    \item \textbf{Gen o genes expresados:} No determinado; se requiere anotación funcional o consulta directa en NCBI.
    \item \textbf{Función biológica:} No determinada; la función se obtiene de la anotación en NCBI si la secuencia corresponde a un gen conocido.
    \item \textbf{Región codificante/no codificante:} No especificada; en NCBI se indica si la secuencia es exón, intrón, promotor, etc.
    \item \textbf{Resultado del alineamiento:}
    \begin{itemize}
        \item \textbf{Score (NW):} $-1234$ (puntuación baja, indica poca similitud global).
        \item \textbf{Identidad:} $85/736$ (12\%).
        \item \textbf{Gaps:} $639/736$ (86\%).
        \item \textbf{Alineamiento:} Ver bloque detallado anterior.
    \end{itemize}
    \item \textbf{Cantidad de HSP (High-scoring Segment Pair):} $1$ (un segmento alineado con puntuación significativa).
    \item \textbf{Cantidad de MSP (Maximal Segment Pair):} $1$ (corresponde al mismo segmento en este caso).
    \item \textbf{Taxonomía (según score más alto):} No determinada; normalmente se clasifica según el organismo con mayor similitud en la base de datos NCBI.
\end{itemize}

\textbf{Interpretación:} El alineamiento muestra baja identidad y requiere muchos gaps, lo que sugiere que las secuencias no están estrechamente relacionadas o sólo comparten regiones cortas similares. Para una interpretación biológica completa, se recomienda consultar los identificadores en NCBI para obtener especie, función, cromosoma y taxonomía.

\begin{figure}[htbp]
    \centering
    \includegraphics[width=0.75\linewidth]{56.png}
    \caption{Alineamiento entre Secuencia 5 y Secuencia 6}
    \label{fig:aln-5-6}
\end{figure}


\end{document}
