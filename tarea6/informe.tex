\documentclass[fleqn,10pt]{article}
\usepackage[utf8]{inputenc}
\usepackage[T1]{fontenc}
\usepackage{graphicx}
\usepackage{geometry}
\usepackage{amsmath,amssymb}
\usepackage{listings}
\usepackage{xcolor}
\usepackage{tcolorbox}
\usepackage{booktabs}
\usepackage{caption}
\usepackage{hyperref}

\geometry{a4paper, margin=1in}

% Estilo para código Python
\definecolor{codegreen}{rgb}{0,0.6,0}
\definecolor{codegray}{rgb}{0.5,0.5,0.5}
\definecolor{codepurple}{rgb}{0.58,0,0.82}
\definecolor{backcolour}{rgb}{0.95,0.95,0.92}

\lstdefinestyle{pythonstyle}{
    backgroundcolor=\color{backcolour},
    commentstyle=\color{codegreen},
    keywordstyle=\color{magenta},
    numberstyle=\tiny\color{codegray},
    stringstyle=\color{codepurple},
    basicstyle=\ttfamily\footnotesize,
    breaklines=true,
    captionpos=b,
    keepspaces=true,
    tabsize=2,
    language=Python,
    showstringspaces=false,
    numbers=left,
    numbersep=5pt,
    frame=single,
    rulecolor=\color{black}
}

% Estilo para la salida de consola o bloques de texto sin resaltar código
\lstdefinestyle{outputstyle}{
    backgroundcolor=\color{backcolour},
    basicstyle=\ttfamily\footnotesize,
    breaklines=true,
    captionpos=b,
    keepspaces=true,
    numbers=none,
    frame=single,
    rulecolor=\color{black},
    showstringspaces=false
}

\begin{document}

% Encabezado principal
\begin{center}
    {\LARGE\bfseries UNIVERSIDAD NACIONAL DE SAN ANTONIO ABAD DEL CUSCO}\\[0.3cm]
    {\Large FACULTAD DE INGENIERÍA ELÉCTRICA, ELECTRÓNICA, INFORMÁTICA Y MECÁNICA}\\[0.3cm]
    {\Large ESCUELA PROFESIONAL DE INGENIERÍA INFORMÁTICA Y DE SISTEMAS}\\[1cm]
    \includegraphics[width=0.25\linewidth]{Escudo_UNSAAC.png}\\[1cm]
    {\Large\bfseries CURSO: BIOINFORMÁTICA}\\[0.3cm]
    {\Large\bfseries TRABAJO: ALINEAMIENTO DE SECUENCIAS}\\[0.3cm]
    {\Large\bfseries PROFESORA: MARIA DEL PILAR VENEGAS VERGARA}\\[0.3cm]
    {\Large\bfseries ALUMNO: EFRAIN VITORINO MARÍN}\\[0.3cm]
    {\Large\bfseries CÓDIGO: 160337}\\[0.3cm]
    {\Large 2025-I}
\end{center}

\bigskip

\section*{¿Qué es el alineamiento de secuencias?}

A continuación se describe la información que se tiene acerca del alineamiento de secuencias:

\subsection*{1. Herramientas Seleccionadas}
\\
EMBOSS Needle (color verde)\\[0.3em]
Biopython (color marrón)

\subsection*{2. Propósito y Descripción de las Herramientas}

\textbf{🔹 EMBOSS Needle}\\[0.3em]
Propósito: Realiza alineamientos globales entre dos secuencias de ADN o proteínas.\\[0.3em]
Algoritmo utilizado: Needleman-Wunsch, que permite comparar secuencias completas desde el primer hasta el último residuo.\\[0.3em]
URL oficial del servicio web:\\ 
\href{https://www.ebi.ac.uk/Tools/psa/emboss_needle/}{https://www.ebi.ac.uk/Tools/psa/emboss\_needle/}\\[0.3em]
¿Por qué global? Porque intenta alinear toda la longitud de ambas secuencias, útil cuando se espera similitud entre ellas en toda su extensión (por ejemplo, genes ortólogos o versiones variantes).\\[0.3em]
Fundamento del algoritmo Needleman-Wunsch:\\[0.3em]
\begin{itemize}
    \item Es un algoritmo de programación dinámica desarrollado en 1970.
    \item Construye una matriz de puntuación comparando todos los pares de caracteres entre ambas secuencias.
    \item Utiliza una función de penalización por huecos (gaps) y recompensa por coincidencias para optimizar la alineación.
    \item Al final, realiza una trazabilidad inversa (traceback) desde la esquina inferior derecha para obtener la alineación óptima global.
\end{itemize}

\textbf{🔸 Biopython}\\[0.3em]
Propósito: Es una biblioteca en Python diseñada para facilitar tareas bioinformáticas como manipulación de secuencias, análisis estructural, lectura de bases de datos y automatización de alineamientos.\\[0.3em]
URL oficial:\\ 
\href{https://biopython.org}{https://biopython.org}\\[0.3em]
Ventaja clave: Permite automatizar tareas que normalmente requerirían múltiples herramientas gráficas o de línea de comandos, integrando todo en scripts reproducibles.

\subsection*{3. Uso Práctico}

\textbf{🧪 A. EMBOSS Needle}\\[0.5em]
Secuencias comparadas:\\[0.5em]

\begin{lstlisting}[style=outputstyle]
shell
Copiar
Editar
>seq1
ATGCTAGCTAGCTAGCTA
>seq2
ATGCTGGCTAGCTAGTTA
\end{lstlisting}

Proceso:\\[0.3em]
Ingresadas en el sitio web EMBOSS Needle\\[0.3em]
Parámetros predeterminados:
\begin{itemize}
    \item Matriz: DNAfull
    \item Gap opening penalty: 10.0
    \item Gap extension penalty: 0.5
\end{itemize}

Resultado resumido:
\begin{lstlisting}[style=outputstyle]
bash
Copiar
Editar
#=======================================
# Aligned_sequences: 2
# 1: seq1
# 2: seq2
# Matrix: DNAfull
# Gap_penalty: 10.0
# Extend_penalty: 0.5
#
# Length: 18
# Identity:      16/18 (88.9%)
# Similarity:    16/18 (88.9%)
# Gaps:           0/18 ( 0.0%)
# Score: 78.0
#=======================================
\end{lstlisting}

Interpretación:\\[0.3em]
Las dos secuencias son altamente similares (89\%).\\
No se introdujeron huecos (gaps), lo cual indica buena conservación estructural.\\
Esto puede implicar relación funcional o evolutiva entre ambas.

\medskip

\textbf{🧪 B. Biopython}\\[0.5em]
Script de ejemplo: Traducción de una secuencia de ADN a proteína
\begin{lstlisting}[style=pythonstyle]
from Bio.Seq import Seq

# Secuencia de ADN
dna = Seq("ATGGCCATTGTAATGGGCCGCTGAAAGGGTGCCCGATAG")

# Traducción a proteína
protein = dna.translate()

print("Proteína traducida:", protein)
\end{lstlisting}

Salida:
\begin{lstlisting}[style=outputstyle]
markdown
Copiar
Editar
Proteína traducida: MAIVMGR*KGAR*
\end{lstlisting}

Análisis:\\[0.3em]
Se traduce automáticamente usando el código genético estándar.\\
El asterisco (*) representa un codón de parada (stop codon).\\
Ideal para ver cómo se expresa un gen a nivel proteico.

\subsection*{4. Conclusiones y Valor Argumentado}

\begin{center}
\begin{tabular}{l l l l}
\toprule
Herramienta & Tipo de Análisis & Caso de Uso & Valor Bioinformático \\
\midrule
EMBOSS Needle & Alineamiento Global & Comparar genes completos & Precisión en relaciones evolutivas o estructurales\\
Biopython & Análisis Programable & Traducción, manipulación de secuencias & Automatización, integración y escalabilidad\\
\bottomrule
\end{tabular}
\end{center}

\medskip

Argumentación final:\\[0.3em]
EMBOSS Needle es ideal cuando se necesita conocer similitudes completas entre genes, como en estudios evolutivos o para validar hipótesis sobre conservación funcional.\\[0.3em]
Biopython, por su parte, permite construir pipelines automatizados para el análisis masivo de datos genómicos, facilitando reproducibilidad y eficiencia en la investigación.

\end{document}
